% Modified from @mikinty 
\usepackage{bm}
\usepackage[margin=1in]{geometry}
\usepackage{enumitem}
\usepackage{fancyhdr}
\usepackage{forest}
\usepackage{float}
\usepackage{graphicx}
\PassOptionsToPackage{hyphens}{url}\usepackage{hyperref} % allow for URL wrapping
\usepackage{mathtools}
\usepackage[standard]{ntheorem}
\usepackage[super]{nth}
\usepackage{pgfplots}
\pgfplotsset{compat=1.18}
\usepackage{physics}
\usepackage{siunitx}
\usepackage{subfig}
\usepackage{soul}
\usepackage{tikz}
\usetikzlibrary{patterns}
\usetikzlibrary{decorations.pathmorphing}
\usepackage{titling}
\usepackage{verbatim}
\usepackage{xspace}

% Custom math shortcuts
\DeclareMathOperator{\Var}{Var}
\DeclareMathOperator{\sign}{sign}
\DeclarePairedDelimiter{\ceil}{\lceil}{\rceil}
\DeclarePairedDelimiter{\floor}{\lfloor}{\rfloor}
\DeclarePairedDelimiter{\paren}{(}{)}
\DeclarePairedDelimiter{\bracken}{[}{]}
\DeclarePairedDelimiter{\bracen}{\{}{\}}
\newcommand{\pa}[1]{\paren*{#1}}
\newcommand{\pbra}[1]{\bracken*{#1}}
\newcommand{\pbrac}[1]{\bracen*{#1}}

% Math symbols
\newcommand{\Z}{Z}
\newcommand{\R}{R}
\newcommand{\Q}{Q}
\newcommand*\conj[1]{\bar{#1}}
\newcommand{\interior}[1]{%
  {\kern0pt#1}^{\mathrm{o}}%
}
\renewcommand{\complement}[1]{%
  {\kern0pt#1}^{\mathrm{c}}%
}
\newcommand{\closure}[1]{%
  \overline{\kern0pt#1}%
}
\newcommand{\limsupn}{\limsup_{n\to\infty}}

% Exercise theorem env
\theoremstyle{break}
\theorembodyfont{\upshape} % No italics in body font
\renewtheorem{definition}{Definition}[section]
\newtheorem{tip}{Tip}[section]
\newtheorem{identity}{Identity}[section]

% Environment Shortcuts
\newcommand{\exercise}[1]{\noindent {\bf Exercise #1} \newline}
\newcommand{\ea}[1]{\begin{enumerate}[label=(\alph*)] #1 \end{enumerate}}

% Shortcuts
\newcommand{\AFSOC}{\text{AFSOC}\xspace}

% Extras
\newcommand{\TODO}{\hl{\textbf{TODO}}\xspace}

% Pathing
\newcommand{\subDir}[1]{chapters/chapter\currChapter/chapter\currChapter-#1}